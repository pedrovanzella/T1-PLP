\documentclass[12pt]{article}

\usepackage[a4paper,width=150mm,top=25mm,bottom=25mm]{geometry}

\usepackage{graphicx,url}

\usepackage[brazil]{babel}   
\usepackage[utf8]{inputenc}  
\usepackage{amsmath}
% UTF-8 encoding is recommended by ShareLaTex

\usepackage[backend=biber]{biblatex}
\addbibresource{relatorio.bib}

\usepackage[activate={true,nocompatibility},final,tracking=true,kerning=true,spacing=true,factor=1100,stretch=10,shrink=10]{microtype}
% activate={true,nocompatibility} - activate protrusion and expansion
% final - enable microtype; use "draft" to disable
% tracking=true, kerning=true, spacing=true - activate these techniques
% factor=1100 - add 10% to the protrusion amount (default is 1000)
% stretch=10, shrink=10 - reduce stretchability/shrinkability (default is 20/20)

\usepackage{float}
\usepackage{listings}

\usepackage{caption}

\usepackage{color}
 
\definecolor{codegreen}{rgb}{0,0.6,0}
\definecolor{codegray}{rgb}{0.5,0.5,0.5}
\definecolor{codepurple}{rgb}{0.58,0,0.82}
\definecolor{backcolour}{rgb}{0.95,0.95,0.92}
 
\lstdefinestyle{mystyle}{
    backgroundcolor=\color{backcolour},   
    commentstyle=\color{codegreen},
    keywordstyle=\color{magenta},
    numberstyle=\tiny\color{codegray},
    stringstyle=\color{codepurple},
    basicstyle=\footnotesize,
    breakatwhitespace=false,         
    breaklines=true,                 
    captionpos=b,                    
    keepspaces=true,                 
    numbers=left,                    
    numbersep=5pt,                  
    showspaces=false,                
    showstringspaces=false,
    showtabs=false,                  
    tabsize=2
}
 
\lstset{style=mystyle}
     
\sloppy

\title{Análise da Linguagem Python}

\author{Pedro Pillon Vanzella}

\date{Agosto de 2015}

\begin{document} 

\maketitle
     
\section{Introdução}\label{sec:introducao}

A linguagem Python foi desenvolvida por Guido Von Rossum em 1991~\cite{venners:2003}. É uma
linguagem de alto nível, de propósito geral e com suporte a múltiplos
paradigmas, como o orientado a objetos, o funcional e o imperativo.

\section{Características}\label{sec:caracteristicas}

A seguir, vamos discutir algumas características da linguagem Python, com foco
na definição sintática e semântica das mesmas.

\subsection{Simplicidade}\label{sec:simplicidade}

\subsection{Expressividade}\label{sec:expressividade}

\subsection{Legibilidade}\label{sec:legibilidade}

\subsection{Redigibilidade}\label{sec:redigibilidade}

\subsection{Capacidade do Tratamento de Excessões}\label{sec:excessoes}


\section{Considerações Finais}


\printbibliography

\end{document}
