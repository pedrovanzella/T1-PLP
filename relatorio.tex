\documentclass[12pt]{article}

\usepackage[a4paper,width=150mm,top=25mm,bottom=25mm]{geometry}

\usepackage{graphicx,url}

\usepackage[brazil]{babel}   
\usepackage[utf8]{inputenc}  
\usepackage{amsmath}
% UTF-8 encoding is recommended by ShareLaTex

\usepackage[backend=biber]{biblatex}
\addbibresource{relatorio.bib}

\usepackage[activate={true,nocompatibility},final,tracking=true,kerning=true,spacing=true,factor=1100,stretch=10,shrink=10]{microtype}
% activate={true,nocompatibility} - activate protrusion and expansion
% final - enable microtype; use "draft" to disable
% tracking=true, kerning=true, spacing=true - activate these techniques
% factor=1100 - add 10% to the protrusion amount (default is 1000)
% stretch=10, shrink=10 - reduce stretchability/shrinkability (default is 20/20)

\usepackage{float}
\usepackage{listings}

\usepackage{caption}

\usepackage{color}
 
\definecolor{codegreen}{rgb}{0,0.6,0}
\definecolor{codegray}{rgb}{0.5,0.5,0.5}
\definecolor{codepurple}{rgb}{0.58,0,0.82}
\definecolor{backcolour}{rgb}{0.95,0.95,0.92}
 
\lstdefinestyle{mystyle}{
    backgroundcolor=\color{backcolour},   
    commentstyle=\color{codegreen},
    keywordstyle=\color{magenta},
    numberstyle=\tiny\color{codegray},
    stringstyle=\color{codepurple},
    basicstyle=\footnotesize,
    breakatwhitespace=false,         
    breaklines=true,                 
    captionpos=b,                    
    keepspaces=true,                 
    numbers=left,                    
    numbersep=5pt,                  
    showspaces=false,                
    showstringspaces=false,
    showtabs=false,                  
    tabsize=2
}
 
\lstset{style=mystyle}
     
\sloppy

\title{Análise da Linguagem Python}

\author{Pedro Pillon Vanzella}

\date{Agosto de 2015}

\begin{document} 

\maketitle
     
\section{Introdução}\label{sec:introducao}

A linguagem Python foi desenvolvida por Guido Von Rossum em 1991~\cite{venners:2003}. É uma
linguagem de alto nível, de propósito geral e com suporte a múltiplos
paradigmas, como o orientado a objetos, o funcional e o imperativo~\cite{Rossum:1995:PRM:869369}.

\section{Características}\label{sec:caracteristicas}

A seguir, vamos discutir algumas características da linguagem Python, com foco
na definição sintática e semântica das mesmas.

\subsection{Simplicidade}\label{sec:simplicidade}

Programas em Python não precisam de nenhum tipo de \textit{boilerplate}, podendo
ser tão curtos quanto necessário~\cite{Rossum:1995:PRM:869369}. Um programa simples, que apenas imprime
``\textit{Hello, world}'', seria:

\begin{lstlisting}
print(`Hello, world')
\end{lstlisting}

A sintaxe de um programa, então, pode ser definida como~\cite{Rossum:1995:PRM:869369}:

\begin{lstlisting}
<comandos>
\end{lstlisting}

Onde \textsf{comandos} é uma lista de \textsf{comando}s, separadas por novas
linhas ou pontos-e-vírgula~\cite{Rossum:1995:PRM:869369}.

\subsection{Expressividade}\label{sec:expressividade}

Uma das características que melhor ilustram a expressividade da linguagem Python
é a atribuição múltipla, que pode ser utilizada para trocar dois valores sem o
uso de uma variável auxiliar~\cite{Rossum:1995:PRM:869369}.

\begin{lstlisting}
<variavel a>, <variavel b> = <variavel b>, <variavel a>
\end{lstlisting}

Aqui vemos duas variáveis sendo atribuídas uma a outra, em apenas um passo.

Por exemplo, tendo duas variáveis, \textsf{a} e \textsf{b}, podemos atribuí-las
uma à outra da seguinte maneira:

\begin{lstlisting}
a, b = b, a
\end{lstlisting}

\subsection{Legibilidade}\label{sec:legibilidade}

A linguagem preza pela legibilidade~\cite{Rossum:1995:PRM:869369}, empregando identação como marca de blocos~\cite{Rossum:1995:PRM:869369}.
Veja, por exemplo, a sintaxe para um bloco \textsf{if-else} com um \textsf{for}
dentro do \textsf{if}:

\begin{lstlisting}
if <condicao>:
    <comando>
    for <variavel> in <lista>:
        <comando>
        <comando>
    <comando-fora-do-for>
else:
    <comando>
\end{lstlisting}

É trivial vermos o que está dentro ou fora do \textsf{for}, apenas olhando para
a identação.

\subsection{Redigibilidade}\label{sec:redigibilidade}

Operações complexas, como filtragem de listas, pode ser feito com \textit{List
  Comprehensions}~\cite{Rossum:1995:PRM:869369}. A sintaxe é a seguinte~\cite{Rossum:1995:PRM:869369}:

\begin{lstlisting}
[<chamada_funcao>(<variavel>) for <variavel> in <lista>]
\end{lstlisting}

Por exemplo, para termos uma lista com o seno de todas os elemetos de outra
lista:

\begin{lstlisting}
[sin(x) for x in list]
\end{lstlisting}

\subsection{Capacidade do Tratamento de Excessões}\label{sec:excessoes}

A linguagem Python possui um bom suporte a excessões~\cite{Rossum:1995:PRM:869369}, com \textsf{try-except}:

\begin{lstlisting}
try:
    <comandos>
except [<exception>]:
    <comandos>
\end{lstlisting}

É possível pegar qualquer excessão omitindo \textsf{exception}, ou pegar apenas
excessões específicas~\cite{Rossum:1995:PRM:869369}.

\section{Considerações Finais}

A linguagem Python é uma excelente escolha para o desenvolvimento de sistemas de
qualquer porte, com excelente suporte aos paradigmas imperativo e orientado a
objetos, bem como um bom suporte ao paradigma funcional, que lhe provê com uma
capacidade superior de expressar construções complexas~\cite{Rossum:1995:PRM:869369}.


\printbibliography

\end{document}
